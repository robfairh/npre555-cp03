\subsection{Conclusions}

\begin{frame}
\frametitle{Conclusions}

	\begin{itemize}
    \item Implemented the kernels to solve the steady-state $SP_3$ equations in the MOOSE-based application Cerberus.
    \item Conducted three exercises whose reference results were known:
    \begin{itemize}
      \item First exercise: Eigenvalue difference of 12 pcm.
      \item Second exercise:
      \begin{itemize}
        \item Eigenvalue difference of 145 pcm.
        \item Pin power distribution within 2\% difference.
      \end{itemize}
      \item Third exercise:
      \begin{itemize}
        \item Eigenvalue difference of 6 pcm.
        \item Assembly power distribution within 1\% difference.
      \end{itemize}
    \end{itemize}
    \item Transport correction is necessary when the scattering higher moments are unknown.
    \item Future work may develop new applications or integrate this application to other physics solvers.
  \end{itemize}
\end{frame}


\subsection{Acknowledgement}

\begin{frame}
\frametitle{Acknowledgement}

This research was performed using funding received from the DOE Office of Nuclear Energy’s University Program (Project 20-19693, DE-NE0008972) ’Evaluation of micro-reactor requirements and performance in an existing well-characterized micro-grid’.

\end{frame}


\subsection{Questions}

\begin{frame}
  \begin{center}
    \Huge{\textbf{Thank you.\\ Questions?}}
  \end{center}
\end{frame}
