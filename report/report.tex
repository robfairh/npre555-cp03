\documentclass[letterpaper]{article}
\usepackage[top=1.0in,bottom=1.0in,left=1.0in,right=1.0in]{geometry}
\usepackage{verbatim}
\usepackage{amssymb}
\usepackage{graphicx}
\usepackage{longtable}
\usepackage{amsfonts}
\usepackage{amsmath}
\usepackage{hyperref}
\usepackage{subfigure}
\usepackage{booktabs}
\def\thesection       {\arabic{section}}
\def\thesubsection     {\thesection.\alph{subsection}}

\author{Roberto E. Fairhurst Agosta
        \\ \href{mailto:ref3@illinois.edu}{\texttt{ref3@illinois.edu}}
}

\title{NPRE 555\\ Computer Project 3}
\begin{document}
%\clearpage
\begin{titlepage}
\maketitle
\thispagestyle{empty}
\end{titlepage}

\section{Introduction}

\section{MOOSE}

\section{Simplified P$_3$: Mathematical Basis}

One dimensional P$_3$ equations
\cite{brantley_simplifiedP3_2000}

\begin{align}
    & \frac{d}{dx} \phi_{1,g} + \Sigma_{t,g} \phi_{0,g} = \sum_{g'=1}^G \Sigma_{s0,g' \rightarrow g} \phi_{0,g'} + \frac{\chi_g}{k_{eff}} \sum_{g'=1}^G \nu\Sigma_{f,g'} \phi_{0,g'} + Q_{0,g}  \label{eq:SP3-0} \\
    & \frac{1}{3} \frac{d}{dx} \phi_{0,g} + \frac{2}{3}\frac{d}{dx}\phi_{2,g} + \Sigma_{t,g} \phi_{1,g} = \sum_{g'=1}^G \Sigma_{s1,g' \rightarrow g} \phi_{1,g'} + Q_{1,g} \label{eq:SP3-1} \\
    & \frac{2}{5} \frac{d}{dx}\phi_{1,g} + \frac{3}{5}\frac{d}{dx}\phi_{3,g} + \Sigma_{t,g} \phi_{2,g} = \sum_{g'=1}^G \Sigma_{s2,g' \rightarrow g} \phi_{2,g'} + Q_{2,g} \label{eq:SP3-2} \\
    & \frac{3}{7}\frac{d}{dx}\phi_{2,g} + \Sigma_{t,g} \phi_{3,g} = \sum_{g'=1}^G \Sigma_{s3,g' \rightarrow g} \phi_{3,g'} + Q_{3,g} \label{eq:P3-3} \\
    \intertext{where}
    & \phi_{n,g} = \mbox{$n^{th}$ moment of the group $g$ neutron flux } [n \cdot cm^{-2} \cdot s^{-1}]  \notag \\
    & \Sigma_{t,g} = \mbox{group $g$ macroscopic total cross-section } [cm^{-1}]  \notag \\
	& \Sigma_{sn,g' \rightarrow g} = \mbox{$n^{th}$ moment of the group $g'$ to group $g$ macroscopic scattering cross-section } [cm^{-1}]  \notag \\
	& \nu\Sigma_{f,g} = \mbox{group $g$ macroscopic production cross-section } [cm^{-1}]  \notag \\
	& \chi_{g} = \mbox{group $g$ fission spectrum } [cm^{-1}]  \notag \\
	& k_{eff} = \mbox{multiplication factor } [-]  \notag \\
	& Q_{n,g} = \mbox{$n^{th}$ group $g$ external neutron source } [n \cdot cm^{-3} \cdot s^{-1}]  \notag \\
	& G = \mbox{number of energy groups } [-].  \notag
\end{align}

Defining the group $g$ "removal" cross-section $\Sigma_{n,g}$, and assuming an isotropic external source and a negligible anisotropic group-to-group scattering \cite{brantley_simplifiedP3_2000}

\begin{align}
	& \Sigma_{n,g} = \Sigma_{t,g} - \Sigma_{sn,g' \rightarrow g} \notag \\
	& Q_{n,g} = 0, \quad n > 0 \notag \\
	& \Sigma_{sn,g' \rightarrow g} = 0, \quad g' \ne g, \quad n > 0 \notag
    \intertext{the P$_3$ equations become}
    & \frac{d}{dx} \phi_{1,g} + \Sigma_{0,g} \phi_{0,g} = \sum_{g'\ne g}^G \Sigma_{s0,g' \rightarrow g} \phi_{0,g'} + \frac{\chi_g}{k_{eff}} \sum_{g'=1}^G \nu\Sigma_{f,g'} \phi_{0,g'} + Q_{0,g}  \label{eq:SP3-0b} \\
    & \frac{1}{3} \frac{d}{dx} \phi_{0,g} + \frac{2}{3}\frac{d}{dx}\phi_{2,g} + \Sigma_{1,g} \phi_{1,g} = 0  \label{eq:SP3-1b} \\
    & \frac{2}{5} \frac{d}{dx}\phi_{1,g} + \frac{3}{5}\frac{d}{dx}\phi_{3,g} + \Sigma_{2,g} \phi_{2,g} = 0  \label{eq:SP3-2b} \\
    & \frac{3}{7}\frac{d}{dx}\phi_{2,g} + \Sigma_{3,g} \phi_{3,g} = 0. \label{eq:SP3-3b}
\end{align}

Reorganizing equations \ref{eq:SP3-1b} and \ref{eq:SP3-3b} allows for obtaining a expression for odd moments of the flux $\phi_{1,g}$ and $\phi_{3,g}$

\begin{align}
    & \phi_{1,g} = -\frac{1}{3 \Sigma_{1,g}} \frac{d}{dx} \left[ \phi_{0,g} + 2 \phi_{2,g} \right] \label{eq:SP3-1c} \\
    & \phi_{3,g} = -\frac{3}{7 \Sigma_{3,g}}\frac{d}{dx}\phi_{2,g}. \label{eq:SP3-3c}
\end{align}

With equations \ref{eq:SP3-1c} and \ref{eq:SP3-3c}, equations \ref{eq:SP3-0b} and \ref{eq:SP3-2b} become

\begin{align}
    & - D_{0,g} \frac{d^2}{dx^2} \left( \phi_{0,g} + 2 \phi_{2,g} \right) + \Sigma_{0,g} \phi_{0,g} = \sum_{g'\ne g}^G \Sigma_{s0,g' \rightarrow g} \phi_{0,g'} + \frac{\chi_g}{k_{eff}} \sum_{g'=1}^G \nu\Sigma_{f,g'} \phi_{0,g'} + Q_{0,g}  \label{eq:SP3-0c} \\
    & - \frac{2}{5} D_{0,g} \frac{d^2}{dx^2} \left( \phi_{0,g} + 2 \phi_{2,g} \right) - D_{2,g} \frac{d^2}{dx^2} \phi_{2,g} + \Sigma_{2,g} \phi_{2,g} = 0  \label{eq:SP3-2c} \\
    \intertext{where}
    & D_{0,g} = \frac{1}{3 \Sigma_{1,g}} \notag \\
    & D_{2,g} = \frac{9}{35 \Sigma_{3,g}} \notag
\end{align}

Introducing the variables $\Phi_{0,g}$ and $\Phi_{2,g}$ and reorganizing equations \ref{eq:SP3-0c} and \ref{eq:SP3-2c} yields

\begin{align}
    & - D_{0,g} \frac{d^2}{dx^2} \Phi_{0,g} + \Sigma_{0,g} \Phi_{0,g} - 2 \Sigma_{0,g} \Phi_{2,g} = S_{0,g} \label{eq:SP3-0d} \\
    & - D_{2,g} \frac{d^2}{dx^2} \Phi_{2,g} + \left( \Sigma_{2,g} + \frac{4}{5} \Sigma_{0,g} \right) \Phi_{2,g} - \frac{2}{5} \Sigma_{0,g} \Phi_{0,g} = -\frac{2}{5} S_{0,g} \label{eq:SP3-2d}
    \intertext{where}
    & \Phi_{0,g} = \phi_{0,g} + 2 \phi_{2,g} \notag \\
    & \Phi_{2,g} = \phi_{2,g} \notag \\
    & S_{0,g} = \sum_{g'\ne g}^G \Sigma_{s0,g' \rightarrow g} \left( \Phi_{0,g'} - 2 \Phi_{2,g'} \right) + \frac{\chi_g}{k_{eff}} \sum_{g'=1}^G \nu\Sigma_{f,g'} \left( \Phi_{0,g'} - 2 \Phi_{2,g'} \right) + Q_{0,g}. \notag
\end{align}

The three-dimensional SP3 equations \cite{gelbard_spherical_1960} replace the second-derivatives in equations \ref{eq:SP3-0d} and \ref{eq:SP3-2d} by the Laplace operator $\Delta$ (See PARCS manual)

\begin{align}
    & - D_{0,g} \Delta \Phi_{0,g} + \Sigma_{0,g} \Phi_{0,g} - 2 \Sigma_{0,g} \Phi_{2,g} = S_{0,g} \label{eq:SP3-0e} \\
    & - D_{2,g} \Delta \Phi_{2,g} + \left( \Sigma_{2,g} + \frac{4}{5} \Sigma_{0,g} \right) \Phi_{2,g} - \frac{2}{5} \Sigma_{0,g} \Phi_{0,g} = -\frac{2}{5} S_{0,g}. \label{eq:SP3-2e}
\end{align}

The Marshak vacuum boundary conditions complete the system of equations

\begin{align}
    & \frac{1}{4} \Phi_{0,g} \pm \frac{1}{2} \hat{n} \cdot J_{0,g} - \frac{3}{16} \Phi_{2,g} = 0 \label{eq:SP3-BC1a} \\
    & - \frac{3}{80} \Phi_{0,g} \pm \frac{1}{2} \hat{n} \cdot J_{2,g} + \frac{21}{80} \Phi_{2,g} = 0 \label{eq:SP3-BC2a}
    \intertext{where}
    & J_{n,g} = -D_{n,g} \nabla \Phi_{n,g}. \notag
\end{align}

Variational formulation

\begin{align}
    & \left< \Phi, \Psi \right> = \int_V \Phi \Psi dV \\
    & \left< \Phi, \Psi \right>_{BC} = \int_S \Phi \Psi dS
    \intertext{where}
    & \Psi = \mbox{test function} \notag \\
    & S = \mbox{boundary surface}. \notag
\end{align}

\begin{align}
    & \left< - D_{0,g} \Delta \Phi_{0,g}, \Psi \right> + \left< \Sigma_{0,g} \Phi_{0,g}, \Psi \right> + \left< - 2 \Sigma_{0,g} \Phi_{2,g}, \Psi \right> + \left< - S_{0,g}, \Psi \right> = 0 \label{eq:SP3-0e} \\
    & \left< - D_{2,g} \Delta \Phi_{2,g}, \Psi \right> + \left< \left( \Sigma_{2,g} + \frac{4}{5} \Sigma_{0,g} \right) \Phi_{2,g}, \Psi \right> + \left< - \frac{2}{5} \Sigma_{0,g} \Phi_{0,g} D_{2,g}, \Psi \right> + \left< \frac{2}{5} S_{0,g}, \Psi \right> = 0. \label{eq:SP3-2e}
\end{align}

By means of the Gauss theorem (?), equations \ref{eq:SP3-0e} and \ref{eq:SP3-2e} become

\begin{align}
    & \left< D_{0,g} \nabla \Phi_{0,g}, \nabla \Psi \right> + \left< - D_{0,g} \nabla \Phi_{0,g}, \Psi \right>_{BC} + \left< \Sigma_{0,g} \Phi_{0,g}, \Psi \right> + \left< - 2 \Sigma_{0,g} \Phi_{2,g}, \Psi \right> + \left< - S_{0,g}, \Psi \right> = 0 \label{eq:SP3-0e} \\
    & \left< D_{2,g} \nabla \Phi_{2,g}, \nabla \Psi \right> + \left< - D_{2,g} \nabla \Phi_{2,g}, \Psi \right>_{BC} + \left< \left( \Sigma_{2,g} + \frac{4}{5} \Sigma_{0,g} \right) \Phi_{2,g}, \Psi \right> + \left< - \frac{2}{5} \Sigma_{0,g} \Phi_{0,g} D_{2,g}, \Psi \right> + \left< \frac{2}{5} S_{0,g}, \Psi \right> = 0. \label{eq:SP3-2e}
\end{align}


% \usepackage{booktabs}
\begin{table}[htbp!]
  \centering
  \caption{.}
  \begin{tabular}{lc}
  \toprule
  Kernel                & Variational form  \\
  \midrule
  P3Diffusion           & $\left< D_{0,g} \nabla \Phi_{0,g}, \nabla \Psi \right>$ \\
  P3SigmaR              &  \\
  P3SigmaCoupled        &  \\
  P3InScatter           &  \\
  P3FissionEigenKernel  &  \\
  \midrule
  Boundary Condition Kernel & Variational form  \\
  \midrule
  -          & - \\
  \bottomrule
  \end{tabular}
  \label{tab:parameters}
\end{table}




% \subsection{Numerical method}

% Through some algebraic manipulation of Equations \ref{eq:P1-0} and \ref{eq:P1-1}, the method obtains the following equation.
% The solver discretizes the equation with the finite difference method.
% Equation \ref{eq:P1-1} combined with the boundary condition equations (Sections \ref{sec:p1-marshak} and \ref{sec:p1-mark}) allow to impose the boundary conditions on the numerical solution.

% \begin{align}
%     & -\frac{d}{dx}\left(\frac{1}{3\Sigma_1} \frac{d}{dx}\phi_0 \right) + \Sigma_0 \phi_0 = q_0
% \end{align}

% \subsection{P$_3$ approximation Marshak boundary condition}
% \label{sec:p3-marshak}

% \begin{align}
%     & \frac{1}{2}\phi_0 (x=0) + \phi_1 (x=0) + \frac{5}{8}\phi_2 (x=0) = 0   \\
%     & -\frac{1}{8}\phi_0 (x=0) + \frac{5}{8}\phi_2 (x=0) + \phi_3 (x=0) = 0  \\
%     & -\frac{1}{2}\phi_0 (x=L) + \phi_1 (x=L) - \frac{5}{8}\phi_2 (x=L) = 0  \\
%     & \frac{1}{8}\phi_0 (x=L) - \frac{5}{8}\phi_2 (x=L) + \phi_3 (x=L) = 0
% \end{align}
% \subsection{P$_3$ approximation Mark boundary condition}
% \label{sec:p3-mark}
% \begin{align}
%     & \frac{1}{2}\phi_0 (x=0) P_0(\mu_0) + \frac{3}{2}\phi_1 (x=0) P_1(\mu_0) + \frac{5}{2}\phi_2 (x=0) P_2(\mu_0) + \frac{7}{2}\phi_3 (x=0) P_3(\mu_0) = 0  \\
%     & \frac{1}{2}\phi_0 (x=0) P_0(\mu_1) + \frac{3}{2}\phi_1 (x=0) P_1(\mu_1) + \frac{5}{2}\phi_2 (x=0) P_2(\mu_1) + \frac{7}{2}\phi_3 (x=0) P_3(\mu_1) = 0  \\
%     & \frac{1}{2}\phi_0 (x=L) P_0(\mu_2) + \frac{3}{2}\phi_1 (x=L) P_1(\mu_2) + \frac{5}{2}\phi_2 (x=L) P_2(\mu_2) + \frac{7}{2}\phi_3 (x=L) P_3(\mu_2) = 0  \\
%     & \frac{1}{2}\phi_0 (x=L) P_0(\mu_3) + \frac{3}{2}\phi_1 (x=L) P_1(\mu_3) + \frac{5}{2}\phi_2 (x=L) P_2(\mu_3) + \frac{7}{2}\phi_3 (x=L) P_3(\mu_3) = 0  \\
%     \intertext{where}
%     & P_0(\mu) = 1    \notag \\
%     & P_1(\mu) = \mu  \notag \\
%     & P_2(\mu) = \frac{1}{2} (3\mu^2-1)    \notag \\
%     & P_3(\mu) = \frac{1}{2} (5\mu^3-\mu)  \notag \\
%     & \mu_{0,1,2,3} = [ 0.86114, 0.33998, -0.33998, -0.86114 ] \notag
% \end{align}
% \subsection{Numerical method}
% Through some algebraic manipulation of Equations \ref{eq:P3-0} to \ref{eq:P3-3}, the method obtains the following equations.
% The solver discretizes the equations with the finite difference method.
% To solution of the coupled system used an explicit solver based on the previous iteration solution, requiring an iterative solver.
% The convergence criteria was an $L_2$-norm of the relative difference between fluxes smaller than $1 \times 10^{-6}$.
% Equations \ref{eq:P3-1} and \ref{eq:P3-3} combined with the boundary condition equations (Sections \ref{sec:p3-marshak} and \ref{sec:p3-mark}) allow to impose the boundary conditions on the numerical solution.
% \begin{align}
%     & -\frac{d}{dx}\left(\frac{1}{3\Sigma_1} \frac{d}{dx}\phi_0 + \frac{2}{3\Sigma_1} \frac{d}{dx}\phi_2 \right) + \Sigma_0 \phi_0 = q_0   \\
%     & -\frac{d}{dx}\left(\frac{2}{3\Sigma_1} \frac{d}{dx}\phi_0 + \left(\frac{4}{3\Sigma_1} + \frac{9}{7\Sigma_3} \right) \frac{d}{dx}\phi_2 \right) + 5 \Sigma_2 \phi_2 = 0
% \end{align}

\section{Results}

\section{Conclusions}

\clearpage
\bibliographystyle{plain}
\bibliography{bibliography}
\end{document}

% \begin{figure}[h!]
%         \centering
%         \includegraphics[width=8cm,height=8cm,keepaspectratio]{AnalyticalSine1.png}
%         \caption{Analytical solution of $P(t)$ for the sinusoidal reactivity in eq. \ref{sine1rho}.}
%         \label{AnalyticSine}
% \end{figure}

% \begin{figure}[h!]
%         \centering
%         \subfigure[Solved using FE.]{\includegraphics[width=8cm,height=8cm,keepaspectratio]{output6FE2b.png}}       
%         \subfigure[Solved using BE.]{\includegraphics[width=8cm,height=8cm,keepaspectratio]{output6BE2b.png}}
%         \caption{Power response for the sinusoidal reactivity in eq. \ref{sine2rho} for 6 Groups, $\Delta t = 1 ms$.}
%         \label{Output6-2b}
% \end{figure}

% \begin{align}
%     0 &= D \nabla^2 \phi - \Sigma_a \phi + \nu \Sigma_f \phi \label{eq:diffusion} \\
%     D &\simeq \frac{1}{3 \Sigma_t}    \notag
%     \intertext{where}
%     D &= \mbox{Diffusion coefficient} \notag \\
%     \Sigma_t &= \mbox{Macroscopic total cross-section} \notag \\
%     \Sigma_a &= \mbox{Macroscopic absorption cross-section} \notag \\
%     \Sigma_f &= \mbox{Macroscopic fission cross-section} \notag \\
%     \nu  &= \mbox{Average number of neutrons born by fission.} \notag \\
% \end{align}
