\documentclass[letterpaper]{article}
\usepackage[top=1.0in,bottom=1.0in,left=1.0in,right=1.0in]{geometry}
\usepackage{verbatim}
\usepackage{amssymb}
\usepackage{graphicx}
\usepackage{longtable}
\usepackage{amsfonts}
\usepackage{amsmath}
\usepackage{hyperref}
\usepackage{subfigure}
\usepackage{booktabs}
\newacronym{ANL}{ANL}{Argonne National Laboratory}
\newacronym{API}{API}{Application Programming Interface}
\newacronym{B4C}{B4C}{boron carbide}
\newacronym{BC}{BC}{boundary condition}
\newacronym{BOC}{BOC}{beginning of the equilibrium cycle}
\newacronym{BSD}{BSD}{Berkeley Software Distribution}
\newacronym{BWR}{BWR}{Boiling Water Reactor}
\newacronym{CAISO}{CAISO}{California ISO}
\newacronym{CAPP}{CAPP}{Core Analyzer for Pebble and Prism type VHTRs}
\newacronym{CEA}{CEA}{Commissariat a l'Energie Atomique}
\newacronym{CFD}{CFD}{computational fluid dynamics}
\newacronym{CO2}{CO$_2$}{carbon dioxide}
\newacronym{CR}{CR}{control rod}
\newacronym{CRP}{CRP}{Coordinated Research Project}
\newacronym{CZP}{CZP}{Cold Zero Power}
\newacronym{DCC}{DCC}{depressurized conduction cool-down}
\newacronym{DOE}{DOE}{Department of Energy}
\newacronym[\glslongpluralkey={degrees of freedom}]{DoF}{DoF}{degree of freedom}
\newacronym{EOC}{EOEC}{end of the equilibrium cycle}
\newacronym{FCEV}{FCEV}{Fuel Cell Electric Vehicle}
\newacronym{FDM}{FDM}{Finite Difference Method}
\newacronym{FEM}{FEM}{Finite Element Method}
\newacronym{FVM}{FVM}{Finite Volume Method}
\newacronym[\glslongpluralkey={greenhouse gases}]{GHG}{GHG}{greenhouse gas}
\newacronym{GRS}{GRS}{Gesellschaft für Anlagen und Reaktorsicherheit}
\newacronym{GT-MHR}{GT-MHR}{Gas Turbine-Modular Helium Reactor}
\newacronym{H2}{H$_2$}{hydrogen}
\newacronym{He}{He}{helium}
\newacronym{HFP}{HFP}{Hot Full Power}
\newacronym{HPCC}{HPCC}{high-pressure conduction cool-down}
\newacronym{HTE}{HTE}{High-Temperature Electrolysis}
\newacronym{HTGR}{HTGR}{High-Temperature Gas-Cooled Reactor}
\newacronym{HTR}{HTR}{High Temperature Reactor}
\newacronym{HTTR}{HTTR}{High-Temperature engineering Test Reactor}
\newacronym{HZDR}{HZDR}{Helmholtz-Zentrum Dresden-Rossendorf}
\newacronym{IAEA}{IAEA}{International Atomic Energy Agency}
\newacronym{icap}{iCAP}{Illinois Climate Action Plan}
\newacronym{INL}{INL}{Idaho National Laboratory}
\newacronym{IPyC}{IPyC}{inner pyrolytic carbon}
\newacronym{JFNK}{JFNK}{Jacobian-Free Newton-Krylov}
\newacronym{KAERI}{KAERI}{Korea Atomic Energy Research Institute}
\newacronym{Keff}{k$_{eff}$}{multiplication factor}
\newacronym{LBP}{LBP}{Lumped Burnable Poison}
\newacronym{LGPL}{LGPL}{Lesser GNU Public License}
\newacronym{LOCA}{LOCA}{loss of coolant accident}
\newacronym{LPCC}{LPCC}{low-pressure conduction cool-down}
\newacronym{LTE}{LTE}{Low-Temperature Electrolysis}
\newacronym{LWR}{LWR}{Light Water Reactor}
\newacronym{MC}{MC}{Monte Carlo}
\newacronym{MHTGR}{MHTGR}{Modular High-Temperature Gas-Cooled Reactor}
\newacronym{MMR}{MMR}{Micro Modular Reactor}
\newacronym{MOC}{MOC}{middle of the equilibrium cycle}
\newacronym{MOX}{MOX}{mixed-oxide}
\newacronym{MOOSE}{MOOSE}{Multi-physics Object-Oriented Simulation Environment}
\newacronym{MPI}{MPI}{Message Passing Interface}
\newacronym{MSR}{MSR}{Molten Salt Reactor}
\newacronym{MTD}{MTD}{Champaign-Urbana Mass Transit District}
\newacronym{NEA}{NEA}{Nuclear Energy Agency}
\newacronym{NEM}{NEM}{Nodal Expansion Method}
\newacronym{NGNP}{NGNP}{Next Generation Nuclear Power}
\newacronym{NRC}{NRC}{Nuclear Regulatory Commission}
\newacronym{NSC}{NSC}{Nuclear Science Committee}
\newacronym{OECD}{OECD}{Organisation for Economic Co-operation and Development}
\newacronym{OPyC}{OPyC}{outer pyrolytic carbon}
\newacronym{ORNL}{ORNL}{Oak Ridge National Laboratory}
\newacronym{OS}{OS}{Operator-Splitting}
\newacronym{PBMR}{PBMR}{Pebble Bed Modular Reactor}
\newacronym{PDE}{PDE}{Partial Differential Equation}
\newacronym{PMR}{PMR}{Prismatic Modular Reactor}
\newacronym{PV}{PV}{photovoltaics}
\newacronym{RPV}{RPV}{Reactor Pressure Vessel}
\newacronym{RSC}{RSC}{Reserve Shutdown Control}
\newacronym{RSD}{RSD}{Relative Standard Deviation}
\newacronym{SD}{SD}{Standard Deviation}
\newacronym{SI}{SI}{Sulfur-Iodine}
\newacronym{SiC}{SiC}{silicon carbide}
\newacronym{SMR}{SMR}{Small Modular Reactor}
\newacronym{SNU}{SNU}{Seoul National University}
\newacronym{SOEC}{SOEC}{Solid Oxide Electrolysis Cells}
\newacronym{SP3}{SP$_3$}{Simplified P$_3$}
\newacronym{TIP}{TIP}{transverse integration procedure}
\newacronym{TRISO}{TRISO}{Tristructural Isotropic}
\newacronym{UIUC}{UIUC}{University of Illinois at Urbana-Champaign}
\newacronym{UNIST}{UNIST}{Ulsan National Institute of Science and Technology}
\newacronym{UK}{UK}{United Kingdom}
\newacronym{UMICH}{UMICH}{University of Michigan}
\newacronym{US}{US}{United States}
\newacronym{USNC}{USNC}{Ultra Safe Nuclear Corporation}
\newacronym{VHTR}{VHTR}{Very High-Temperature Gas-Cooled Reactor}
%\newacronym{<++>}{<++>}{<++>}
%\newacronym{<++>}{<++>}{<++>}

\def\thesection       {\arabic{section}}
\def\thesubsection     {\thesection.\alph{subsection}}

\author{Roberto E. Fairhurst Agosta
        \\ \href{mailto:ref3@illinois.edu}{\texttt{ref3@illinois.edu}}
}

\title{NPRE 555\\ Computer Project 3}
\begin{document}
%\clearpage
\begin{titlepage}
\maketitle
\thispagestyle{empty}
\end{titlepage}

\section{Introduction}

\section{MOOSE}

MOOSE \cite{gaston_moose_2009} is a computational framework that supports engineering analysis applications.
In a nuclear reactor, several partial differential equations describe the physical behavior.
These equations are typically nonlinear, and they are often coupled to each other.
\gls{MOOSE} targets such systems and solves them in a fully coupled manner.

% more details about MOOSE
\gls{MOOSE} is an open-source FEM framework.
The framework itself relies on LibMesh \cite{kirk_libmesh_2006} and PetSc \cite{balay_petsc_2016} for solving nonlinear equations.
MOOSE applications define weak forms of the governing equations and modularize the physics expressions into "Kernels."
Kernels are C++ classes containing methods for computing the residual and Jacobian contributions of individual pieces of the governing equations.
\gls{MOOSE} and LibMesh translate them into residual and Jacobian functions.
These functions become inputs into PetSc solution routines.

\gls{MOOSE} utilizes the \gls{JFNK} method \cite{knoll_jacobian-free_2004} mathematical structure \cite{gaston_moose_2009}.
\gls{JFNK} methods are synergistic combinations of Newton-type methods for superlinearly convergence of nonlinear equations and Krylov subspace methods for solving the Newton correction equations.
The Jacobian-vector product links the two methods.
JFNK methods compute such products approximately without forming and storing the elements of the true Jacobian.
The ability to perform a Newton iteration without forming the Jacobian gives JFNK methods potential for application throughout problems governed by nonlinear partial differential equations.

All the software built on the MOOSE framework shares the same \gls{API}.
The applications, by default, utilize monolithic and implicit methods \cite{lindsay_introduction_2018}.
This feature facilitates relatively easy coupling between different phenomena and allows for great flexibility, even with a great variance in time scales \cite{novak_pronghorn_2018}.
Additionally, the framework and its applications use \gls{MPI} for parallel communication and allow deployment on massively-parallel cluster-computing platforms.


\section{Simplified P$_3$: Mathematical Basis}

One dimensional P$_3$ equations
\cite{brantley_simplifiedP3_2000}

\begin{align}
    & \frac{d}{dx} \phi_{1,g} + \Sigma_{t,g} \phi_{0,g} = \sum_{g'=1}^G \Sigma_{s0,g' \rightarrow g} \phi_{0,g'} + \frac{\chi_g}{k_{eff}} \sum_{g'=1}^G \nu\Sigma_{f,g'} \phi_{0,g'} + Q_{0,g}  \label{eq:SP3-0} \\
    & \frac{1}{3} \frac{d}{dx} \phi_{0,g} + \frac{2}{3}\frac{d}{dx}\phi_{2,g} + \Sigma_{t,g} \phi_{1,g} = \sum_{g'=1}^G \Sigma_{s1,g' \rightarrow g} \phi_{1,g'} + Q_{1,g} \label{eq:SP3-1} \\
    & \frac{2}{5} \frac{d}{dx}\phi_{1,g} + \frac{3}{5}\frac{d}{dx}\phi_{3,g} + \Sigma_{t,g} \phi_{2,g} = \sum_{g'=1}^G \Sigma_{s2,g' \rightarrow g} \phi_{2,g'} + Q_{2,g} \label{eq:SP3-2} \\
    & \frac{3}{7}\frac{d}{dx}\phi_{2,g} + \Sigma_{t,g} \phi_{3,g} = \sum_{g'=1}^G \Sigma_{s3,g' \rightarrow g} \phi_{3,g'} + Q_{3,g} \label{eq:P3-3} \\
    \intertext{where}
    & \phi_{n,g} = \mbox{$n^{th}$ moment of the group $g$ neutron flux } [n \cdot cm^{-2} \cdot s^{-1}]  \notag \\
    & \Sigma_{t,g} = \mbox{group $g$ macroscopic total cross-section } [cm^{-1}]  \notag \\
	& \Sigma_{sn,g' \rightarrow g} = \mbox{$n^{th}$ moment of the group $g'$ to group $g$ macroscopic scattering cross-section } [cm^{-1}]  \notag \\
	& \nu\Sigma_{f,g} = \mbox{group $g$ macroscopic production cross-section } [cm^{-1}]  \notag \\
	& \chi_{g} = \mbox{group $g$ fission spectrum } [cm^{-1}]  \notag \\
	& k_{eff} = \mbox{multiplication factor } [-]  \notag \\
	& Q_{n,g} = \mbox{$n^{th}$ group $g$ external neutron source } [n \cdot cm^{-3} \cdot s^{-1}]  \notag \\
	& G = \mbox{number of energy groups } [-].  \notag
\end{align}

Defining the group $g$ "removal" cross-section $\Sigma_{n,g}$, and assuming an isotropic external source and a negligible anisotropic group-to-group scattering \cite{brantley_simplifiedP3_2000}

\begin{align}
	& \Sigma_{n,g} = \Sigma_{t,g} - \Sigma_{sn,g' \rightarrow g} \notag \\
	& Q_{n,g} = 0, \quad n > 0 \notag \\
	& \Sigma_{sn,g' \rightarrow g} = 0, \quad g' \ne g, \quad n > 0 \notag
    \intertext{the P$_3$ equations become}
    & \frac{d}{dx} \phi_{1,g} + \Sigma_{0,g} \phi_{0,g} = \sum_{g'\ne g}^G \Sigma_{s0,g' \rightarrow g} \phi_{0,g'} + \frac{\chi_g}{k_{eff}} \sum_{g'=1}^G \nu\Sigma_{f,g'} \phi_{0,g'} + Q_{0,g}  \label{eq:SP3-0b} \\
    & \frac{1}{3} \frac{d}{dx} \phi_{0,g} + \frac{2}{3}\frac{d}{dx}\phi_{2,g} + \Sigma_{1,g} \phi_{1,g} = 0  \label{eq:SP3-1b} \\
    & \frac{2}{5} \frac{d}{dx}\phi_{1,g} + \frac{3}{5}\frac{d}{dx}\phi_{3,g} + \Sigma_{2,g} \phi_{2,g} = 0  \label{eq:SP3-2b} \\
    & \frac{3}{7}\frac{d}{dx}\phi_{2,g} + \Sigma_{3,g} \phi_{3,g} = 0. \label{eq:SP3-3b}
\end{align}

Reorganizing equations \ref{eq:SP3-1b} and \ref{eq:SP3-3b} allows for obtaining a expression for odd moments of the flux $\phi_{1,g}$ and $\phi_{3,g}$

\begin{align}
    & \phi_{1,g} = -\frac{1}{3 \Sigma_{1,g}} \frac{d}{dx} \left[ \phi_{0,g} + 2 \phi_{2,g} \right] \label{eq:SP3-1c} \\
    & \phi_{3,g} = -\frac{3}{7 \Sigma_{3,g}}\frac{d}{dx}\phi_{2,g}. \label{eq:SP3-3c}
\end{align}

With equations \ref{eq:SP3-1c} and \ref{eq:SP3-3c}, equations \ref{eq:SP3-0b} and \ref{eq:SP3-2b} become

\begin{align}
    & - D_{0,g} \frac{d^2}{dx^2} \left( \phi_{0,g} + 2 \phi_{2,g} \right) + \Sigma_{0,g} \phi_{0,g} = \sum_{g'\ne g}^G \Sigma_{s0,g' \rightarrow g} \phi_{0,g'} + \frac{\chi_g}{k_{eff}} \sum_{g'=1}^G \nu\Sigma_{f,g'} \phi_{0,g'} + Q_{0,g}  \label{eq:SP3-0c} \\
    & - \frac{2}{5} D_{0,g} \frac{d^2}{dx^2} \left( \phi_{0,g} + 2 \phi_{2,g} \right) - D_{2,g} \frac{d^2}{dx^2} \phi_{2,g} + \Sigma_{2,g} \phi_{2,g} = 0  \label{eq:SP3-2c} \\
    \intertext{where}
    & D_{0,g} = \frac{1}{3 \Sigma_{1,g}} \notag \\
    & D_{2,g} = \frac{9}{35 \Sigma_{3,g}} \notag
\end{align}

Introducing the variables $\Phi_{0,g}$ and $\Phi_{2,g}$ and reorganizing equations \ref{eq:SP3-0c} and \ref{eq:SP3-2c} yields

\begin{align}
    & - D_{0,g} \frac{d^2}{dx^2} \Phi_{0,g} + \Sigma_{0,g} \Phi_{0,g} - 2 \Sigma_{0,g} \Phi_{2,g} = S_{0,g} \label{eq:SP3-0d} \\
    & - D_{2,g} \frac{d^2}{dx^2} \Phi_{2,g} + \left( \Sigma_{2,g} + \frac{4}{5} \Sigma_{0,g} \right) \Phi_{2,g} - \frac{2}{5} \Sigma_{0,g} \Phi_{0,g} = -\frac{2}{5} S_{0,g} \label{eq:SP3-2d}
    \intertext{where}
    & \Phi_{0,g} = \phi_{0,g} + 2 \phi_{2,g} \notag \\
    & \Phi_{2,g} = \phi_{2,g} \notag \\
    & S_{0,g} = \sum_{g'\ne g}^G \Sigma_{s0,g' \rightarrow g} \left( \Phi_{0,g'} - 2 \Phi_{2,g'} \right) + \frac{\chi_g}{k_{eff}} \sum_{g'=1}^G \nu\Sigma_{f,g'} \left( \Phi_{0,g'} - 2 \Phi_{2,g'} \right) + Q_{0,g}. \notag
\end{align}

The three-dimensional SP3 equations \cite{gelbard_spherical_1960} replace the second-derivatives in equations \ref{eq:SP3-0d} and \ref{eq:SP3-2d} by the Laplace operator $\Delta$ (See PARCS manual)

\begin{align}
    & - D_{0,g} \Delta \Phi_{0,g} + \Sigma_{0,g} \Phi_{0,g} - 2 \Sigma_{0,g} \Phi_{2,g} = S_{0,g} \label{eq:SP3-0e} \\
    & - D_{2,g} \Delta \Phi_{2,g} + \left( \Sigma_{2,g} + \frac{4}{5} \Sigma_{0,g} \right) \Phi_{2,g} - \frac{2}{5} \Sigma_{0,g} \Phi_{0,g} = -\frac{2}{5} S_{0,g}. \label{eq:SP3-2e}
\end{align}

The Marshak vacuum boundary conditions complete the system of equations

\begin{align}
    & \frac{1}{4} \Phi_{0,g} \pm \frac{1}{2} \hat{n} \cdot J_{0,g} - \frac{3}{16} \Phi_{2,g} = 0 \label{eq:SP3-BC1a} \\
    & - \frac{3}{80} \Phi_{0,g} \pm \frac{1}{2} \hat{n} \cdot J_{2,g} + \frac{21}{80} \Phi_{2,g} = 0 \label{eq:SP3-BC2a}
    \intertext{where}
    & J_{n,g} = -D_{n,g} \nabla \Phi_{n,g}. \notag
\end{align}

Variational formulation

\begin{align}
    & \left< \Phi, \Psi \right> = \int_V \Phi \Psi dV \\
    & \left< \Phi, \Psi \right>_{BC} = \int_S \Phi \Psi dS
    \intertext{where}
    & \Psi = \mbox{test function} \notag \\
    & S = \mbox{boundary surface}. \notag
\end{align}

\begin{align}
    & \left< - D_{0,g} \Delta \Phi_{0,g}, \Psi \right> + \left< \Sigma_{0,g} \Phi_{0,g}, \Psi \right> + \left< - 2 \Sigma_{0,g} \Phi_{2,g}, \Psi \right> + \left< - S_{0,g}, \Psi \right> = 0 \label{eq:SP3-0e} \\
    & \left< - D_{2,g} \Delta \Phi_{2,g}, \Psi \right> + \left< \left( \Sigma_{2,g} + \frac{4}{5} \Sigma_{0,g} \right) \Phi_{2,g}, \Psi \right> + \left< - \frac{2}{5} \Sigma_{0,g} \Phi_{0,g} D_{2,g}, \Psi \right> + \left< \frac{2}{5} S_{0,g}, \Psi \right> = 0. \label{eq:SP3-2e}
\end{align}

By means of the Gauss theorem (?), equations \ref{eq:SP3-0e} and \ref{eq:SP3-2e} become

\begin{align}
    & \left< D_{0,g} \nabla \Phi_{0,g}, \nabla \Psi \right> + \left< - D_{0,g} \nabla \Phi_{0,g}, \Psi \right>_{BC} + \left< \Sigma_{0,g} \Phi_{0,g}, \Psi \right> + \left< - 2 \Sigma_{0,g} \Phi_{2,g}, \Psi \right> \\ &+ \left< - \sum_{g'\ne g}^G \Sigma_{s0,g' \rightarrow g} \left( \Phi_{0,g'} - 2 \Phi_{2,g'} \right), \Psi \right> + \left< - \frac{\chi_g}{k_{eff}} \sum_{g'=1}^G \nu\Sigma_{f,g'} \left( \Phi_{0,g'} - 2 \Phi_{2,g'} \right), \Psi \right> + \left< - Q_{0,g}, \Psi \right> = 0 \label{eq:SP3-0e} \\
    & \left< D_{2,g} \nabla \Phi_{2,g}, \nabla \Psi \right> + \left< - D_{2,g} \nabla \Phi_{2,g}, \Psi \right>_{BC} + \left< \left( \Sigma_{2,g} + \frac{4}{5} \Sigma_{0,g} \right) \Phi_{2,g}, \Psi \right> + \left< - \frac{2}{5} \Sigma_{0,g} \Phi_{0,g}, \Psi \right> \\ &+ \left< \frac{2}{5} \sum_{g'\ne g}^G \Sigma_{s0,g' \rightarrow g} \left( \Phi_{0,g'} - 2 \Phi_{2,g'} \right), \Psi \right> + \left< \frac{2}{5} \frac{\chi_g}{k_{eff}} \sum_{g'=1}^G \nu\Sigma_{f,g'} \left( \Phi_{0,g'} - 2 \Phi_{2,g'} \right), \Psi \right> + \left< \frac{2}{5} Q_{0,g}, \Psi \right> = 0. \label{eq:SP3-2e}
\end{align}

Vacuum BCs kernels?

\begin{align}
    & \left< - D_{0,g} \nabla \Phi_{0,g}, \Psi \right>_{BC} = \left< -\frac{1}{2} \Phi_{0,g} + \frac{3}{4} \Phi_{2,g}, \Psi \right>_{BC} \\
    & \left< - D_{2,g} \nabla \Phi_{2,g}, \Psi \right>_{BC} = \left< \frac{3}{40} \Phi_{0,g} - \frac{21}{40} \Phi_{2,g}, \Psi \right>_{BC} \\
\end{align}

% \usepackage{booktabs}
\begin{table}[htbp!]
  \centering
  \caption{.}
  \begin{tabular}{lcc}
  \toprule
  % multicolumn Variational form
  Kernel                & Equation A & Equation B \\
  \midrule
  P3Diffusion           & $\left< D_{0,g} \nabla \Phi_{0,g}, \nabla \Psi \right>$ & $\left< D_{2,g} \nabla \Phi_{2,g}, \nabla \Psi \right>$ \\
  P3SigmaR              & $\left< \Sigma_{0,g} \Phi_{0,g}, \Psi \right>$ & $\left< \left( \Sigma_{2,g} + \frac{4}{5} \Sigma_{0,g} \right) \Phi_{2,g}, \Psi \right>$ \\
  P3SigmaCoupled        & $\left< - 2 \Sigma_{0,g} \Phi_{2,g}, \Psi \right>$ & $\left< - \frac{2}{5} \Sigma_{0,g} \Phi_{0,g}, \Psi \right>$ \\
  P3InScatter           & $\left< - \sum_{g'\ne g}^G \Sigma_{s0,g' \rightarrow g} \left( \Phi_{0,g'} - 2 \Phi_{2,g'} \right), \Psi \right>$ & $\left< \frac{2}{5} \sum_{g'\ne g}^G \Sigma_{s0,g' \rightarrow g} \left( \Phi_{0,g'} - 2 \Phi_{2,g'} \right), \Psi \right>$ \\
  P3FissionEigenKernel  & $\left< - \frac{\chi_g}{k_{eff}} \sum_{g'=1}^G \nu\Sigma_{f,g'} \left( \Phi_{0,g'} - 2 \Phi_{2,g'} \right), \Psi \right>$ & $\left< \frac{2}{5} \frac{\chi_g}{k_{eff}} \sum_{g'=1}^G \nu\Sigma_{f,g'} \left( \Phi_{0,g'} - 2 \Phi_{2,g'} \right), \Psi \right>$ \\
  \midrule
  Boundary Condition Kernel & Equation A & Equation B \\
  \midrule
  Vacuum          & $\left< -\frac{1}{2} \Phi_{0,g} + \frac{3}{4} \Phi_{2,g}, \Psi \right>_{BC}$ & $\left< \frac{3}{40} \Phi_{0,g} - \frac{21}{40} \Phi_{2,g}, \Psi \right>_{BC}$ \\
  \bottomrule
  \end{tabular}
  \label{tab:parameters}
\end{table}




% \subsection{Numerical method}

% Through some algebraic manipulation of Equations \ref{eq:P1-0} and \ref{eq:P1-1}, the method obtains the following equation.
% The solver discretizes the equation with the finite difference method.
% Equation \ref{eq:P1-1} combined with the boundary condition equations (Sections \ref{sec:p1-marshak} and \ref{sec:p1-mark}) allow to impose the boundary conditions on the numerical solution.

% \begin{align}
%     & -\frac{d}{dx}\left(\frac{1}{3\Sigma_1} \frac{d}{dx}\phi_0 \right) + \Sigma_0 \phi_0 = q_0
% \end{align}

% \subsection{P$_3$ approximation Marshak boundary condition}
% \label{sec:p3-marshak}

% \begin{align}
%     & \frac{1}{2}\phi_0 (x=0) + \phi_1 (x=0) + \frac{5}{8}\phi_2 (x=0) = 0   \\
%     & -\frac{1}{8}\phi_0 (x=0) + \frac{5}{8}\phi_2 (x=0) + \phi_3 (x=0) = 0  \\
%     & -\frac{1}{2}\phi_0 (x=L) + \phi_1 (x=L) - \frac{5}{8}\phi_2 (x=L) = 0  \\
%     & \frac{1}{8}\phi_0 (x=L) - \frac{5}{8}\phi_2 (x=L) + \phi_3 (x=L) = 0
% \end{align}
% \subsection{P$_3$ approximation Mark boundary condition}
% \label{sec:p3-mark}
% \begin{align}
%     & \frac{1}{2}\phi_0 (x=0) P_0(\mu_0) + \frac{3}{2}\phi_1 (x=0) P_1(\mu_0) + \frac{5}{2}\phi_2 (x=0) P_2(\mu_0) + \frac{7}{2}\phi_3 (x=0) P_3(\mu_0) = 0  \\
%     & \frac{1}{2}\phi_0 (x=0) P_0(\mu_1) + \frac{3}{2}\phi_1 (x=0) P_1(\mu_1) + \frac{5}{2}\phi_2 (x=0) P_2(\mu_1) + \frac{7}{2}\phi_3 (x=0) P_3(\mu_1) = 0  \\
%     & \frac{1}{2}\phi_0 (x=L) P_0(\mu_2) + \frac{3}{2}\phi_1 (x=L) P_1(\mu_2) + \frac{5}{2}\phi_2 (x=L) P_2(\mu_2) + \frac{7}{2}\phi_3 (x=L) P_3(\mu_2) = 0  \\
%     & \frac{1}{2}\phi_0 (x=L) P_0(\mu_3) + \frac{3}{2}\phi_1 (x=L) P_1(\mu_3) + \frac{5}{2}\phi_2 (x=L) P_2(\mu_3) + \frac{7}{2}\phi_3 (x=L) P_3(\mu_3) = 0  \\
%     \intertext{where}
%     & P_0(\mu) = 1    \notag \\
%     & P_1(\mu) = \mu  \notag \\
%     & P_2(\mu) = \frac{1}{2} (3\mu^2-1)    \notag \\
%     & P_3(\mu) = \frac{1}{2} (5\mu^3-\mu)  \notag \\
%     & \mu_{0,1,2,3} = [ 0.86114, 0.33998, -0.33998, -0.86114 ] \notag
% \end{align}
% \subsection{Numerical method}
% Through some algebraic manipulation of Equations \ref{eq:P3-0} to \ref{eq:P3-3}, the method obtains the following equations.
% The solver discretizes the equations with the finite difference method.
% To solution of the coupled system used an explicit solver based on the previous iteration solution, requiring an iterative solver.
% The convergence criteria was an $L_2$-norm of the relative difference between fluxes smaller than $1 \times 10^{-6}$.
% Equations \ref{eq:P3-1} and \ref{eq:P3-3} combined with the boundary condition equations (Sections \ref{sec:p3-marshak} and \ref{sec:p3-mark}) allow to impose the boundary conditions on the numerical solution.
% \begin{align}
%     & -\frac{d}{dx}\left(\frac{1}{3\Sigma_1} \frac{d}{dx}\phi_0 + \frac{2}{3\Sigma_1} \frac{d}{dx}\phi_2 \right) + \Sigma_0 \phi_0 = q_0   \\
%     & -\frac{d}{dx}\left(\frac{2}{3\Sigma_1} \frac{d}{dx}\phi_0 + \left(\frac{4}{3\Sigma_1} + \frac{9}{7\Sigma_3} \right) \frac{d}{dx}\phi_2 \right) + 5 \Sigma_2 \phi_2 = 0
% \end{align}

\section{Results}

\section{Conclusions}

\clearpage
\bibliographystyle{plain}
\bibliography{bibliography}
\end{document}

% \begin{figure}[h!]
%         \centering
%         \includegraphics[width=8cm,height=8cm,keepaspectratio]{AnalyticalSine1.png}
%         \caption{Analytical solution of $P(t)$ for the sinusoidal reactivity in eq. \ref{sine1rho}.}
%         \label{AnalyticSine}
% \end{figure}

% \begin{figure}[h!]
%         \centering
%         \subfigure[Solved using FE.]{\includegraphics[width=8cm,height=8cm,keepaspectratio]{output6FE2b.png}}       
%         \subfigure[Solved using BE.]{\includegraphics[width=8cm,height=8cm,keepaspectratio]{output6BE2b.png}}
%         \caption{Power response for the sinusoidal reactivity in eq. \ref{sine2rho} for 6 Groups, $\Delta t = 1 ms$.}
%         \label{Output6-2b}
% \end{figure}

% \begin{align}
%     0 &= D \nabla^2 \phi - \Sigma_a \phi + \nu \Sigma_f \phi \label{eq:diffusion} \\
%     D &\simeq \frac{1}{3 \Sigma_t}    \notag
%     \intertext{where}
%     D &= \mbox{Diffusion coefficient} \notag \\
%     \Sigma_t &= \mbox{Macroscopic total cross-section} \notag \\
%     \Sigma_a &= \mbox{Macroscopic absorption cross-section} \notag \\
%     \Sigma_f &= \mbox{Macroscopic fission cross-section} \notag \\
%     \nu  &= \mbox{Average number of neutrons born by fission.} \notag \\
% \end{align}
