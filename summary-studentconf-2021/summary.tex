\documentclass{anstrans}
%%%%%%%%%%%%%%%%%%%%%%%%%%%%%%%%%%%
\title{Implementation of the SP3 equations in a MOOSE-based application}
\author{Roberto E. Fairhurst Agosta, Kathryn D. Huff}

\institute{
University of Illinois at Urbana-Champaign, Dept. of Nuclear, Plasma, and Radiological Engineering\\
ref3@illinois.edu
}

%%%% packages and definitions (optional)
\usepackage{graphicx} % allows inclusion of graphics
\usepackage{booktabs} % nice rules (thick lines) for tables
\usepackage{microtype} % improves typography for PDF
\usepackage{xspace}
\usepackage{tabularx}
\usepackage{floatrow}
\usepackage{subcaption}
\usepackage{enumitem}
\usepackage{placeins}
\usepackage{amsmath}
\usepackage[acronym,toc]{glossaries}
\include{acros}
\makeglossaries

\usepackage[printwatermark]{xwatermark}
\usepackage{xcolor}
\usepackage{graphicx}
\usepackage{lipsum}

\newcommand{\SN}{S$_N$}
\renewcommand{\vec}[1]{\bm{#1}} %vector is bold italic
\newcommand{\vd}{\bm{\cdot}} % slightly bold vector dot
\newcommand{\grad}{\vec{\nabla}} % gradient
\newcommand{\ud}{\mathop{}\!\mathrm{d}} % upright derivative symbol

\newcolumntype{c}{>{\hsize=.56\hsize}X}
\newcolumntype{b}{>{\hsize=.7\hsize}X}
\newcolumntype{s}{>{\hsize=.74\hsize}X}
\newcolumntype{f}{>{\hsize=.1\hsize}X}
\newcolumntype{a}{>{\hsize=.45\hsize}X}
%\usepackage[pagestyles]{titlesec}
%\titleformat*{\subsection}{\normalfont}
%\titleformat{\section}{\bfseries}{Item \thesection.\ }{0pt}{}

%\newwatermark[allpages,color=gray!50,angle=45,scale=3,xpos=0,ypos=0]{DRAFT}

\begin{document}
%%%%%%%%%%%%%%%%%%%%%%%%%%%%%%%%%%%%%%%%%%%%%%%%%%%%%%%%%%%%%%%%%%%%%%%%%%%%%%%%

\section{Abstract}
% \textit{

% abstract goes here

% }

\section{Introduction}

% what am I doing


% lit - review: the SPN approximation
The $P_N$ method \cite{davidson_neutron_1957} discretizes the transport equation by expanding the angular dependence of the neutron flux in spherical harmonics, considering up to order $N$ polynomials.
If $N \rightarrow \infty$, the solution of the $P_N$ equations tends to the exact transport solution.
In three-dimensional geometries, the number of $P_N$ equations is proportional to $(N+1)^2$, whereas, in one-dimensional planar geometries, the number of $P_N$ equations is $(N+1)$.
Gelbard \cite{gelbard_spherical_1960} proposed the $SP_N$ approximation by replacing the second derivatives in the one-dimensional planar $P_N$ equations with three-dimensional Laplacian operators.
This approximation considerably reduces the number of equations conserving a reasonable accuracy.


Although Gelbard proposed the $SP_N$ approximation in the 60s, the method did not have a strong theoretical basis until the 90s with Pomraning \cite{pomraning_asymptotic_1993}, Brantley, and Larsen's \cite{brantley_simplifiedP3_2000} contribution.


The disadvantage of the $SP_N$ approximation is that the solution does not usually converge to the true transport solution as $N \rightarrow \infty$.

The $SP_N$ equations yield asymptotic solutions of the transport equation in a physical regime in which diffusion theory is the leading-order approximation.



However, several results \cite{mui_modified_1987} \cite{beckert_development_2007} \cite{fliscounakis_potential_2012} \cite{ryu_finite_2013} \cite{khosravi_mirzaee_reactor_2019} show that the $SP_N$ approximation yields more accurate solutions than the diffusion approximation with considerably less computational expense than the discrete ordinates ($S_N$) method.


% why is SP3 preferred over diffusion



% lit - review: software using the SPN approximation
Currently, different software uses the SP3 approximation to solve the neutron transport equation, such as SCOPE2 \cite{tatsumi_object-oriented_2002}, PARCS \cite{downar_parcs_2004}, DYN3D \cite{beckert_development_2007}, SIMULATE-5 \cite{bahadir_studsviks_2009}, and COCAGNE \cite{fliscounakis_potential_2012}.


% something about MOOSE - maybe move it up so I have a better motivation


\section{Methodology}

% Equations P3
Davidson \cite{davidson_neutron_1957} defined the one dimensional multi-group P$_3$ equations 

\begin{align}
    & \frac{d}{dx} \phi_{1,g} + \Sigma_{t,g} \phi_{0,g} = \sum_{g'=1}^G \Sigma_{s0,g' \rightarrow g} \phi_{0,g'} + \frac{\chi_g}{k_{eff}} \sum_{g'=1}^G \nu\Sigma_{f,g'} \phi_{0,g'} \notag \\ & \quad \quad \quad \quad \quad \quad \quad + Q_{0,g}  \label{eq:P3-0} \\
    & \frac{1}{3} \frac{d}{dx} \phi_{0,g} + \frac{2}{3}\frac{d}{dx}\phi_{2,g} + \Sigma_{t,g} \phi_{1,g} = \sum_{g'=1}^G \Sigma_{s1,g' \rightarrow g} \phi_{1,g'} + Q_{1,g} \label{eq:P3-1} \\
    & \frac{2}{5} \frac{d}{dx}\phi_{1,g} + \frac{3}{5}\frac{d}{dx}\phi_{3,g} + \Sigma_{t,g} \phi_{2,g} = \sum_{g'=1}^G \Sigma_{s2,g' \rightarrow g} \phi_{2,g'} + Q_{2,g} \label{eq:P3-2} \\
    & \frac{3}{7}\frac{d}{dx}\phi_{2,g} + \Sigma_{t,g} \phi_{3,g} = \sum_{g'=1}^G \Sigma_{s3,g' \rightarrow g} \phi_{3,g'} + Q_{3,g} \label{eq:P3-3} \\
    \intertext{where}
    & \phi_{n,g} = \mbox{$n^{th}$ moment of group $g$ neutron flux}  \notag \\ %  } [n \cdot cm^{-2} \cdot s^{-1}]
    & \Sigma_{t,g} = \mbox{group $g$ macroscopic total cross-section}  \notag \\ %  } [cm^{-1}]
	& \Sigma_{sn,g' \rightarrow g} = \mbox{$n^{th}$ moment of the group $g'$ to group $g$} \notag \\
	& \mbox{macroscopic scattering cross-section}  \notag \\ %  } [cm^{-1}]
	& \nu\Sigma_{f,g} = \mbox{group $g$ macroscopic production cross-section}  \notag \\ %  } [cm^{-1}]
	& \chi_{g} = \mbox{group $g$ fission spectrum}  \notag \\ %  } [cm^{-1}]
	& k_{eff} = \mbox{multiplication factor}  \notag \\ %  } [-]
	& Q_{n,g} = \mbox{$n^{th}$ moment of group $g$ external neutron source}  \notag \\  % } [n \cdot cm^{-3} \cdot s^{-1}]
	& G = \mbox{number of energy groups}.  \notag % } [-]
\end{align}

Assuming an isotropic external source and a negligible anisotropic group-to-group scattering \cite{brantley_simplifiedP3_2000}
\begin{align}
	& Q_{n,g} = 0, \quad n > 0 \notag \\
	& \Sigma_{sn,g' \rightarrow g} = 0, \quad g' \ne g, \quad n > 0 \notag
\end{align}

\noindent
simplifies equations \ref{eq:P3-1} and \ref{eq:P3-3}, allowing to express the odd moments of the flux $\phi_{1,g}$ and $\phi_{3,g}$ as functions of the even moments $\phi_{0,g}$ and $\phi_{2,g}$.
Introducing $\phi_{1,g}$ and $\phi_{3,g}$ into equations \ref{eq:P3-0} and \ref{eq:P3-2} reduces the system of equations from four to two equations.
Introducing the variables $\Phi_{0,g}$ and $\Phi_{2,g}$, reorganizing the equations, and replacing the second derivatives by Laplacian operators \cite{gelbard_spherical_1960} yields the SP$_3$ equations

% Equations SP3
\begin{align}
    & - D_{0,g} \Delta \Phi_{0,g} + \Sigma_{0,g} \Phi_{0,g} - 2 \Sigma_{0,g} \Phi_{2,g} = S_{0,g} \label{eq:SP3-0e} \\
    & - D_{2,g} \Delta \Phi_{2,g} + \left( \Sigma_{2,g} + \frac{4}{5} \Sigma_{0,g} \right) \Phi_{2,g} - \frac{2}{5} \Sigma_{0,g} \Phi_{0,g} = -\frac{2}{5} S_{0,g}. \label{eq:SP3-2e}
    \intertext{where}
	& \Sigma_{n,g} = \Sigma_{t,g} - \Sigma_{sn,g' \rightarrow g} \notag \\
    & \Phi_{0,g} = \phi_{0,g} + 2 \phi_{2,g} \notag \\
    & \Phi_{2,g} = \phi_{2,g} \notag \\
    & D_{0,g} = \frac{1}{3 \Sigma_{1,g}} \notag \\
    & D_{2,g} = \frac{9}{35 \Sigma_{3,g}} \notag \\
    & S_{0,g} = \sum_{g'\ne g}^G \Sigma_{s0,g' \rightarrow g} \left( \Phi_{0,g'} - 2 \Phi_{2,g'} \right) \notag \\
    & \quad \quad + \frac{\chi_g}{k_{eff}} \sum_{g'=1}^G \nu\Sigma_{f,g'} \left( \Phi_{0,g'} - 2 \Phi_{2,g'} \right) + Q_{0,g}. \notag
\end{align}



\section{Results}

\subsection{C5 MOX Benchmark}

OECD/NEACRP-L-336 C5 MOX fuel benchmark \cite{cavarec_benchmark_1994}

% why is SP3 preferred over diffusion: maybe move to intro ??
MOX fuel assemblies have higher thermal absorption and fission cross-sections than UO$_2$ fuel assemblies.
Consequently, the thermal flux is lower while the power production higher.
Modeling these characteristics of reactors using MOX fuel using the diffusion approximation may be challenging.

% benchmark specification ?

Figure \ref{fig:bench1}

Figures \ref{fig:bench2} and \ref{fig:bench3} display the geometry of each type of fuel assembly.

\begin{figure}[htbp!] %or H 
    \centering
    \includegraphics[width=0.95\linewidth]{figures/bench-config.png}
    \hfill
    \caption{C5 MOX benchmark configuration. \textit{R} correponds to the reflector region. Image reproduced from \cite{capilla_applications_2009}.}
    \label{fig:bench1}
\end{figure}

\begin{figure}[htbp!] %or H 
    \centering
    \includegraphics[width=0.95\linewidth]{figures/bench-config2.png}
    \hfill
    \caption{Structure of the UO$_2$ assembly. Image reproduced from \cite{capilla_applications_2009}.}
    \label{fig:bench2}
\end{figure}

\begin{figure}[htbp!] %or H 
    \centering
    \includegraphics[width=0.95\linewidth]{figures/bench-config3.png}
    \hfill
    \caption{Structure of the MOX assembly. Image reproduced from \cite{capilla_applications_2009}.}
    \label{fig:bench3}
\end{figure}

% % \usepackage{booktabs}
% \begin{table}[]
% 	\centering
%     \caption{\gls{GGE}, \gls{DGE}, and CO$_2$ produced.}
%     \label{tab:meth}
% 	\begin{tabular}{l|lll}
% 	\toprule
% 	                 & Hydrogen & Gasoline    & Diesel      \\
% 	\midrule
% 	GGE              & 1 kg     & 1 gallon    & 0.88 gallon \\
% 	DGE              & 1.13 kg  & 1.13 gallon & 1 gallon    \\
%     CO$_2$ produced  & -        & 19.64 lbs   & 22.38 lbs   \\
%     \bottomrule
% 	\end{tabular}
% \end{table}

\section{Conclusions}



\section{Acknowledgements}

% Not sure about this
Roberto E. Fairhurst Agosta and Prof. Huff are supported by the \gls{NRC} Faculty Development Program (award NRC-HQ-84-14-G-0054 Program B).
Prof. Huff is also supported by the Blue Waters sustained-petascale computing project supported by the National Science Foundation (awards OCI-0725070 and ACI-1238993) and the state of Illinois, the DOE ARPA-E MEITNER Program (award DE-AR0000983), the DOE H2@Scale Program (award), and the International Institute for Carbon Neutral Energy Research (WPI-I2CNER), sponsored by the Japanese Ministry of Education, Culture, Sports, Science and Technology.

%%%%%%%%%%%%%%%%%%%%%%%%%%%%%%%%%%%%%%%%%%%%%%%%%%%%%%%%%%%%%%%%%%%%%%%%%%%%%%%%
\bibliographystyle{ans}
\bibliography{bibliography}
\end{document}

% \begin{figure}[htbp!] %or H 
%     \centering
%     \includegraphics[width=0.95\linewidth]{figures/radial-layout.png}
%     \hfill
%     \caption{Core radial layout. Image reproduced from \cite{oecd_nea_benchmark_2017}.}
%     \label{fig:radial}
% \end{figure}

% \begin{align}
%     \frac{1}{v_g}\frac{\partial}{\partial t} \phi_g &= \nabla \cdot D_g
%     \nabla \phi_g - \Sigma_g^r \phi_g \sum_{g \ne g'}^G
%     \Sigma_{g'\rightarrow g}^s \phi_{g'} \label{eq:diffusion}
%     \intertext{where}
%     C_i &= \mbox{concentration of delayed neutron precursors} \notag \\
%     &\phantom{{}=1} \mbox{in precursor group $i$}.
% \end{align}
