\documentclass{anstrans}
%%%%%%%%%%%%%%%%%%%%%%%%%%%%%%%%%%%
\title{Implementation of the SP3 equations in a MOOSE-based application}
\author{Roberto E. Fairhurst Agosta, Kathryn D. Huff}

\institute{
University of Illinois at Urbana-Champaign, Dept. of Nuclear, Plasma, and Radiological Engineering\\
ref3@illinois.edu
}

%%%% packages and definitions (optional)
\usepackage{graphicx} % allows inclusion of graphics
\usepackage{booktabs} % nice rules (thick lines) for tables
\usepackage{microtype} % improves typography for PDF
\usepackage{xspace}
\usepackage{tabularx}
\usepackage{caption}
\usepackage{floatrow}
\usepackage{subcaption}
\usepackage{enumitem}
\usepackage{placeins}
\usepackage{amsmath}
\usepackage[acronym,toc]{glossaries}
\include{acros}
\makeglossaries

\usepackage[printwatermark]{xwatermark}
\usepackage{xcolor}
\usepackage{graphicx}
\usepackage{lipsum}

\newcommand{\SN}{S$_N$}
\renewcommand{\vec}[1]{\bm{#1}} %vector is bold italic
\newcommand{\vd}{\bm{\cdot}} % slightly bold vector dot
\newcommand{\grad}{\vec{\nabla}} % gradient
\newcommand{\ud}{\mathop{}\!\mathrm{d}} % upright derivative symbol

\newcolumntype{c}{>{\hsize=.56\hsize}X}
\newcolumntype{b}{>{\hsize=.7\hsize}X}
\newcolumntype{s}{>{\hsize=.74\hsize}X}
\newcolumntype{f}{>{\hsize=.1\hsize}X}
\newcolumntype{a}{>{\hsize=.45\hsize}X}
%\usepackage[pagestyles]{titlesec}
%\titleformat*{\subsection}{\normalfont}
%\titleformat{\section}{\bfseries}{Item \thesection.\ }{0pt}{}

%\newwatermark[allpages,color=gray!50,angle=45,scale=3,xpos=0,ypos=0]{DRAFT}

\begin{document}
%%%%%%%%%%%%%%%%%%%%%%%%%%%%%%%%%%%%%%%%%%%%%%%%%%%%%%%%%%%%%%%%%%%%%%%%%%%%%%%%

\section{Abstract}
% \textit{

% abstract goes here

% }

\section{Introduction}

% what am I doing
This work presents the implementation of the $SP_3$ equations \cite{gelbard_spherical_1960} in a \gls{MOOSE}-based application.
% MOOSE
MOOSE \cite{gaston_moose_2009} is a computational framework that supports engineering analysis applications.
In a nuclear reactor, several partial differential equations describe its physical behavior.
These equations are typically nonlinear, and they are often coupled to each other.
MOOSE targets such systems solving them in a fully coupled manner.

MOOSE is an open-source \gls{FEM} framework.
The framework itself relies on LibMesh \cite{kirk_libmesh_2006} and PetSc \cite{balay_petsc_2016} for solving nonlinear equations.
MOOSE-based applications define weak forms of the governing equations and modularize the physics expressions into "kernels."
Kernels are C++ classes containing methods for computing the residual and Jacobian contributions of individual pieces of the governing equations.
MOOSE and LibMesh translate them into residual and Jacobian functions.
These functions become inputs into PetSc solution routines.

All the software built on the MOOSE framework shares the same \gls{API}, facilitating relatively easy coupling between different phenomena.
While the $SP_3$ equations solve the neutronics in a nuclear reactor, other applications may solve the thermal-fluids and given they share the same API, their integration is straightforward.
Additionally, the framework and its applications use \gls{MPI} for parallel communication and allow deployment on massively-parallel cluster-computing platforms.

% lit - review: the SPN approximation
The $P_N$ method \cite{davidson_neutron_1957} discretizes the transport equation by expanding the angular dependence of the neutron flux in spherical harmonics, considering up to order $N$ polynomials.
If $N \rightarrow \infty$, the solution of the $P_N$ equations tends to the exact transport solution.
In three-dimensional geometries, the number of $P_N$ equations is proportional to $(N+1)^2$, whereas, in one-dimensional planar geometries, the number of $P_N$ equations is $(N+1)$.
Gelbard \cite{gelbard_spherical_1960} proposed the $SP_N$ approximation by replacing the second derivatives in the one-dimensional planar $P_N$ equations with three-dimensional Laplacian operators.
This approximation considerably reduces the number of equations conserving a reasonable accuracy.
Capilla et al. \cite{capilla_applications_2009} conducted an extension of the C5 \gls{MOX} fuel Benchmark \cite{cavarec_benchmark_1994} introduced by Brantley and Larsen \cite{brantley_simplifiedP3_2000} comparing the $P_3$ and $SP_3$ methods, the difference between results being less than 40 pcm.

The $SP_N$ approximation has the disadvantage that the solution does not usually converge to the true transport solution as $N \rightarrow \infty$.
Additionally, the theoretical basis of the formulation of $SP_N$ approximation was weak.
For these reasons, the method did not gain widespread use until the 2000s, when thanks to Pomraning \cite{pomraning_asymptotic_1993}, Brantley, and Larsen's \cite{brantley_simplifiedP3_2000} contribution, the method gained a stronger theoretical basis.
% Although Gelbard proposed the $SP_N$ approximation in the 60s, the method did not have a strong theoretical basis until the 90s with Pomraning \cite{pomraning_asymptotic_1993}, Brantley, and Larsen's \cite{brantley_simplifiedP3_2000} contribution.
% Additionally, the $SP_N$ approximation has the disadvantage that the solution does not usually converge to the true transport solution as $N \rightarrow \infty$.

In practice, the $SP_N$ equations are most accurate for diffusive problems or for problems in which the solution behaves nearly one-dimensionally and has weak tangential derivatives at material interfaces.
For problems that have strong multidimensional transport effects, such as voids, streaming regions, or geometrically complex regions, the $SP_N$ solutions are less accurate \cite{downar_parcs_2004}.
However, several results show that the $SP_N$ approximation yields more accurate solutions than the diffusion approximation \cite{mui_modified_1987} \cite{beckert_development_2007} \cite{fliscounakis_potential_2012} \cite{ryu_finite_2013} \cite{khosravi_mirzaee_reactor_2019} with considerably less computational expense than the discrete ordinates ($S_N$) method \cite{brantley_simplifiedP3_2000}.
% why is SP3 preferred over diffusion
For example, the $SP_3$ approximation is preferable over the diffusion approximation for modeling reactors using MOX/UO$_2$ fuel assemblies.
MOX fuel assemblies have higher thermal absorption and fission cross-sections than UO$_2$ fuel assemblies, and consequently, their thermal flux is lower while their power production higher.
Modeling these characteristics using the diffusion approximation may be challenging \cite{brantley_simplifiedP3_2000} \cite{capilla_applications_2009}.

% lit - review: software using the SPN approximation
The $SP_3$ approximation gained popularity throughout the last couple of decades and currently, different software uses it to solve the neutron transport equation. Some of those software are SCOPE2 \cite{tatsumi_object-oriented_2002}, PARCS \cite{downar_parcs_2004}, DYN3D \cite{beckert_development_2007}, SIMULATE-5 \cite{bahadir_studsviks_2009}, and COCAGNE \cite{fliscounakis_potential_2012}.


\section{Methodology}

This section describes the methodology followed for solving the equations.
% Equations P3
Davidson \cite{davidson_neutron_1957} defined the one dimensional multi-group P$_N$ equations.
For $N=3$, the equations become

\begin{align}
    & \frac{d}{dx} \phi_{1,g} + \Sigma_{t,g} \phi_{0,g} = \sum_{g'=1}^G \Sigma_{s0,g' \rightarrow g} \phi_{0,g'} + \frac{\chi_g}{k_{eff}} \sum_{g'=1}^G \nu\Sigma_{f,g'} \phi_{0,g'} \notag \\ & \quad \quad \quad \quad \quad \quad \quad + Q_{0,g}  \label{eq:P3-0} \\
    & \frac{1}{3} \frac{d}{dx} \phi_{0,g} + \frac{2}{3}\frac{d}{dx}\phi_{2,g} + \Sigma_{t,g} \phi_{1,g} = \sum_{g'=1}^G \Sigma_{s1,g' \rightarrow g} \phi_{1,g'} + Q_{1,g} \label{eq:P3-1} \\
    & \frac{2}{5} \frac{d}{dx}\phi_{1,g} + \frac{3}{5}\frac{d}{dx}\phi_{3,g} + \Sigma_{t,g} \phi_{2,g} = \sum_{g'=1}^G \Sigma_{s2,g' \rightarrow g} \phi_{2,g'} + Q_{2,g} \label{eq:P3-2} \\
    & \frac{3}{7}\frac{d}{dx}\phi_{2,g} + \Sigma_{t,g} \phi_{3,g} = \sum_{g'=1}^G \Sigma_{s3,g' \rightarrow g} \phi_{3,g'} + Q_{3,g} \label{eq:P3-3} \\
    \intertext{where}
    & \phi_{n,g} = \mbox{$n^{th}$ moment of group $g$ neutron flux}  \notag \\ %  } [n \cdot cm^{-2} \cdot s^{-1}]
    & \Sigma_{t,g} = \mbox{group $g$ macroscopic total cross-section}  \notag \\ %  } [cm^{-1}]
	& \Sigma_{sn,g' \rightarrow g} = \mbox{$n^{th}$ moment of the group $g'$ to group $g$} \notag \\
	& \mbox{macroscopic scattering cross-section}  \notag \\ %  } [cm^{-1}]
	& \nu\Sigma_{f,g} = \mbox{group $g$ macroscopic production cross-section}  \notag \\ %  } [cm^{-1}]
	& \chi_{g} = \mbox{group $g$ fission spectrum}  \notag \\ %  } [cm^{-1}]
	& k_{eff} = \mbox{multiplication factor}  \notag \\ %  } [-]
	& Q_{n,g} = \mbox{$n^{th}$ moment of group $g$ external neutron source}  \notag \\  % } [n \cdot cm^{-3} \cdot s^{-1}]
	& G = \mbox{number of energy groups}.  \notag % } [-]
\end{align}

Assuming an isotropic external source and a negligible anisotropic group-to-group scattering \cite{brantley_simplifiedP3_2000}
\begin{align}
	& Q_{n,g} = 0, \quad n > 0 \notag \\
	& \Sigma_{sn,g' \rightarrow g} = 0, \quad g' \ne g, \quad n > 0 \notag
\end{align}

\noindent
simplifies equations \ref{eq:P3-1} and \ref{eq:P3-3}, allowing to express the odd moments of the flux $\phi_{1,g}$ and $\phi_{3,g}$ as functions of the even moments $\phi_{0,g}$ and $\phi_{2,g}$.
Introducing $\phi_{1,g}$ and $\phi_{3,g}$ into equations \ref{eq:P3-0} and \ref{eq:P3-2} reduces the system of equations from four to two equations.
Introducing the variables $\Phi_{0,g}$ and $\Phi_{2,g}$, reorganizing the equations, and replacing the second derivatives by Laplacian operators \cite{gelbard_spherical_1960} yields the SP$_3$ equations \cite{beckert_development_2007}

% Equations SP3
\begin{align}
    & - D_{0,g} \Delta \Phi_{0,g} + \Sigma_{0,g} \Phi_{0,g} - 2 \Sigma_{0,g} \Phi_{2,g} = S_{0,g} \label{eq:SP3-0e} \\
    & - D_{2,g} \Delta \Phi_{2,g} + \left( \Sigma_{2,g} + \frac{4}{5} \Sigma_{0,g} \right) \Phi_{2,g} - \frac{2}{5} \Sigma_{0,g} \Phi_{0,g} = -\frac{2}{5} S_{0,g} \label{eq:SP3-2e}
    \intertext{where}
	& \Sigma_{n,g} = \Sigma_{t,g} - \Sigma_{sn,g' \rightarrow g} \notag \\
    & \Phi_{0,g} = \phi_{0,g} + 2 \phi_{2,g} \notag \\
    & \Phi_{2,g} = \phi_{2,g} \notag \\
    & D_{0,g} = \frac{1}{3 \Sigma_{1,g}} \notag \\
    & D_{2,g} = \frac{9}{35 \Sigma_{3,g}} \notag \\
    & S_{0,g} = \sum_{g'\ne g}^G \Sigma_{s0,g' \rightarrow g} \left( \Phi_{0,g'} - 2 \Phi_{2,g'} \right) \notag \\
    & \quad \quad + \frac{\chi_g}{k_{eff}} \sum_{g'=1}^G \nu\Sigma_{f,g'} \left( \Phi_{0,g'} - 2 \Phi_{2,g'} \right) + Q_{0,g}. \notag
\end{align}

% BCs
The Marshak-like vacuum \glspl{BC} complete the system of equations \cite{beckert_development_2007}

\begin{align}
    & \frac{1}{4} \Phi_{0,g} \pm \frac{1}{2} \hat{n} \cdot J_{0,g} - \frac{3}{16} \Phi_{2,g} = 0 \label{eq:SP3-BC1a} \\
    - & \frac{3}{80} \Phi_{0,g} \pm \frac{1}{2} \hat{n} \cdot J_{2,g} + \frac{21}{80} \Phi_{2,g} = 0 \label{eq:SP3-BC2a}
    \intertext{where}
    & J_{n,g} = -D_{n,g} \nabla \Phi_{n,g}. \notag
\end{align}

Finally, multiplying equations \ref{eq:SP3-0e} and \ref{eq:SP3-2e} by a test function and integrating over the domain allows for obtaining the weak form of the equations that we modularized into kernels in the MOOSE-based application.
For brevity, we will not display the derivation of the kernels here.
Such procedure is standard in weighted residual methods and can be found in \cite{ryu_finite_2013} and any finite elements book \cite{quarteroni_numerical_1994}.


\section{Results}

\subsection{One group two-dimensional problem}

Problem defined by \cite{brantley_simplifiedP3_2000}
Figure \ref{fig:2D}

\begin{figure}[htbp!] %or H 
    \centering
    \includegraphics[width=0.95\linewidth]{figures/brantley-larsen.png}
    \hfill
    \caption{Image reproduced from \cite{brantley_simplifiedP3_2000}.}
    \label{fig:2D}
\end{figure}

% \usepackage{booktabs}
\begin{table}[htbp!]
	\centering
	% \caption{Cross-sections. Values expressed in $cm^{-1}$.}
	\caption{Cross-sections.}
	\label{tab:cross-sections}
	\begin{tabular}{llll}
	\toprule
	Material	& $\Sigma_t$ & $\Sigma_{s0}$ & $\nu\Sigma_f$ \\
	\midrule
	M 			& 1.00		& 0.93			& 0.00			\\
	F 			& 1.50		& 1.35			& 0.24			\\
	\bottomrule
	\end{tabular}
\end{table}

\begin{table}[htbp!]
	\centering
	\caption{.}
	\label{tab:keff}
	\begin{tabular}{llll}
	\toprule
							& Reference & MOOSE 	& $\Delta_{\rho}$ [pcm]	\\
	\midrule
	Normal 					& 0.79862	& 0.79854	& 12					\\
	\bottomrule
	\end{tabular}
\end{table}

\subsection{C5 MOX Benchmark}

% intro to the benchmark
OECD/NEACRP-L-336 C5 MOX fuel benchmark \cite{cavarec_benchmark_1994}

The \gls{OECD}/\gls{NEA} developed this benchmark to carry out validation of methods and identify their strengths, limitations, and accuracy, and to suggest needs for method development.

The definition of the original benchmark can be found in \cite{cavarec_benchmark_1994}.

% benchmark specification
Two types of fuel assembly (MOX and UO$_2$) and a reflector comprise the core, shown in Figure \ref{fig:bench1}.
Due to the problem's symmetry, the model included only a quarter of the core.
Each fuel assembly consists of a 17 $\times$ 17 array of squared pin cells, as displayed in Figures \ref{fig:bench2} and \ref{fig:bench3}.
The dimensions of each pin cell are 1.26 $\times$ 1.26 cm, being 21.42 $\times$ 21.42 cm the length of each assembly, and 128.52 $\times$ 128.52 cm.

The benchmark \cite{cavarec_benchmark_1994} specifies the cross-sections, which have a two-energy group structure.

% add reference
Gmsh \cite{geuzaine_gmsh_2009}

%
Diagonal transport correction - i.e. use the transport cross-section to calculate the diffusion coefficient
\begin{align}
  & D = \frac{1}{3 \Sigma_{tr}} \\
  & \Sigma_{tr} = \Sigma_t - \bar{\mu} \Sigma_{s0}
  \intertext{where}
  & \Sigma_{tr} = \mbox{transport cross-section} \notag \\
  & \bar{\mu} = \mbox{average cosine deviation angle.} \notag
\end{align}

\begin{figure}[htbp!] %or H 
    \centering
    \includegraphics[width=0.95\linewidth]{figures/bench-config.png}
    \hfill
    \caption{C5 MOX benchmark configuration. \textit{R} correponds to the reflector region. Image reproduced from \cite{capilla_applications_2009}.}
    \label{fig:bench1}
\end{figure}

\begin{figure}[htbp!] %or H 
    \centering
    \includegraphics[width=0.95\linewidth]{figures/bench-config2.png}
    \hfill
    \caption{Structure of the UO$_2$ assembly. Image reproduced from \cite{capilla_applications_2009}.}
    \label{fig:bench2}
\end{figure}

\begin{figure}[htbp!] %or H 
    \centering
    \includegraphics[width=0.95\linewidth]{figures/bench-config3.png}
    \hfill
    \caption{Structure of the MOX assembly. Image reproduced from \cite{capilla_applications_2009}.}
    \label{fig:bench3}
\end{figure}

Normal \cite{capilla_applications_2009}
Diagonal transport correction \cite{cavarec_benchmark_1994}
When no anisotropic component of the scattering cross section is available the 'diagonal transport correction' is used to compensate the lack of data for the scattering anisotropy \cite{cavarec_benchmark_1994}

Table \ref{tab:keff} compares between the eigenvalue obtained with the SP3 solver and the reference
\begin{align}
  & \Delta_\rho = \left| \rho_{SP_3} - \rho_{Ref} \right| = \left| \frac{k_{SP_3}-1}{k_{SP_3}} - \frac{k_{Ref}-1}{k_{Ref}} \right| = \left| \frac{k_{SP_3}-k_{Ref}}{k_{SP_3} k_{Ref}} \right| \label{eq:delta-rho} \\
  \intertext{where}
  & k_{SP_3} = \mbox{eigenvalue obtained with SP3 solver} \notag \\
  & k_{Ref} = \mbox{ reference eigenvalue.} \notag
\end{align}

% \usepackage{booktabs}
\begin{table}[]
	\centering
	\caption{.}
	\label{tab:keff}
	\begin{tabular}{llll}
	\toprule
							& Reference & MOOSE 	& $\Delta_{\rho}$ [pcm]	\\
	\midrule
	Normal 					& 0.96969	& 0.97106	& 145					\\
	Transport correction 	& 0.93755	& 0.93792	& 43					\\
	\bottomrule
	\end{tabular}
\end{table}

The benchmark specifies the power distribution pin-by-pin.
To simplify the display of the results, Figure \ref{fig:power-distrib} presents the power distribution in each assembly.

\begin{figure}[htbp!] %or H 
    \centering
    \includegraphics[width=0.95\linewidth]{figures/distrib.png}
    \hfill
    \caption{Power distribution in the different assemblies. Top: power distribution. Bottom: relative difference to reference values expressed in \%.}
    \label{fig:power-distrib}
\end{figure}

% Discussion ?

Diagonal transport correction:
C5 MOX Benchmark \cite{cavarec_benchmark_1994}, PARCS \cite{downar_parcs_2004}, DYN3D \cite{beckert_development_2007}

\section{Conclusions}



\section{Acknowledgements}

% Not sure about this
Roberto E. Fairhurst Agosta and Prof. Huff are supported by the \gls{NRC} Faculty Development Program (award NRC-HQ-84-14-G-0054 Program B).
Prof. Huff is also supported by the Blue Waters sustained-petascale computing project supported by the National Science Foundation (awards OCI-0725070 and ACI-1238993) and the state of Illinois, the DOE ARPA-E MEITNER Program (award DE-AR0000983), the DOE H2@Scale Program (award), and the International Institute for Carbon Neutral Energy Research (WPI-I2CNER), sponsored by the Japanese Ministry of Education, Culture, Sports, Science and Technology.

%%%%%%%%%%%%%%%%%%%%%%%%%%%%%%%%%%%%%%%%%%%%%%%%%%%%%%%%%%%%%%%%%%%%%%%%%%%%%%%%
\bibliographystyle{ans}
\bibliography{bibliography}
\end{document}

% \begin{figure}[htbp!] %or H 
%     \centering
%     \includegraphics[width=0.95\linewidth]{figures/radial-layout.png}
%     \hfill
%     \caption{Core radial layout. Image reproduced from \cite{oecd_nea_benchmark_2017}.}
%     \label{fig:radial}
% \end{figure}

% \begin{align}
%     \frac{1}{v_g}\frac{\partial}{\partial t} \phi_g &= \nabla \cdot D_g
%     \nabla \phi_g - \Sigma_g^r \phi_g \sum_{g \ne g'}^G
%     \Sigma_{g'\rightarrow g}^s \phi_{g'} \label{eq:diffusion}
%     \intertext{where}
%     C_i &= \mbox{concentration of delayed neutron precursors} \notag \\
%     &\phantom{{}=1} \mbox{in precursor group $i$}.
% \end{align}
